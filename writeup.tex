\documentclass{article}
\usepackage[cm]{fullpage}
\title{Pentesting}
%\subtitle{ECE478 Network Security}
\author{Dean Johnson \& Emily Dunham}
\linespread{1.5}

\begin{document}

\maketitle
\section{What's pentesting? }
    Penetration testing is the occupation of those security professionals who
attempt to gain access to a system using only the tools and knowledge available
to a malicious attacker, but with the system's owner's consent
\cite{whatispentest}. 

\section{How can pentesting be done legally?}

The two key differences between pentesters and malicious attackers are that
the pentester attacks a system only with its owner's consent, and the
malicious attacker exploits discovered vulnerabilities in a way that
harms or disregards the wellbeing of the system's owner. 

A legal pentest starts out different from a malicious attack in that the
tester gets the system's owner's permission first. The pentester can
optionally be given additional information about the system, to help guide
their attacks by avoiding time that would otherwise be wasted in gaining those
basic facts. 

\section{Jobs in the field} 

\section{Targets/Clients}

\section{Tools \& Techniques}

\subsection{Social engineering}

\subsection{SQL injection}

\subsection{XSS}

\subsection{JS injection}

\subsection{Privilege escalation}

\subsection{Known, unpatched vulnerabilities}

\section{What results might be found? }

\section{What actions would be taken by the company as a result?}

\begin{thebibliography}{1}
    \bibitem{whatispentest} https://www.sans.org/reading-room/analysts-program/PenetrationTesting-June06
    
    \bibitem{securitycurrent}
        http://www.securitycurrent.com/en/writers/mark-rasch/legal-issues-in-penetration-testing
    \bibitem{insidetrust} http://insidetrust.blogspot.com/2011/01/penetration-testing-permission.html
\end{thebibliography}
\end{document}
