\documentclass{article}
\usepackage{fullpage}
\title{Pentesting\\ECE478 Network Security}
\author{Dean Johnson \& Emily Dunham}
\linespread{1.5}

\begin{document}

\maketitle
\section{What's pentesting? }
    Penetration testing is the occupation of those security professionals who
attempt to gain access to a system using only the tools and knowledge available
to a malicious attacker, but with the system's owner's consent
\cite{whatispentest}. 

Pentesting includes automatic vulnerability scanning, manual testing, and a
variety of other techniques that mimic those which would be used by malicious
attackers. 

\subsection{Types of tests\cite{definition}}

{\bf Targeted testing} is a form of ``white box" testing in which testers work with
the client's internal teams who maintain the product being tested. This can
help simulate the level of knowledge and access available to an attacker such
as a disgruntled employee. 

{\bf External testing} explores how far an external attacker could get into a
client's systems by performing an attack from outside the company's network
and systems, trying to break in. External testing usually involves giving the
tester some special knowledge about the systems, which might not be available
to all malicious attackers. 

{\bf Internal testing} seeks ways that a disgruntled employee could escalate
the priveliges of standard system access in order to harm the customer's
systems. 

In {\bf blind testing}, the tester is given minimal information about the
network being tested. They will know the company's name, which can be used to
explore tech blogs, social networking sites, press releases, and any other
disclosures of information -- however, the tester knows no more than a
malicious attacker from outside the company would. 

For all of the above types of tests, it's assumed that all of the relevant
personnel inside the client company will be aware that the tests are being
performed. However, in {\bf double blind testing}, the client's teams who run
the systems being tested are not informed that the test is taking place.
Double blind is the only form of testing which can accurately predict the
client's response to a real attack, since in all other forms of testing people
are warned about the test and will be more vigilant during the testing period. 

\section{Pentesting Clients}

Any company or organization who could be harmed by a malicious attack on their
computer systems can benefit from penetration testing. Some of the most common
pentesting clients include banks, e-commerce sites, governments, corporations,
and militaries. 

Additionally, any company that builds or sells products that make security
claims, such as firewall programs or operating system code, can save itself
the embarrassment of some public vulnerability disclosures by having its
product pentested before release. Open source projects related to network
security, such as the router firmware project Tomato, generally get some level
of security testing from ordinary attacks against their users' deployments. 

\section{How can pentesting be done legally?}

The two key differences between pentesters and malicious attackers are that
the pentester attacks a system only with its owner's consent, and the
malicious attacker exploits discovered vulnerabilities in a way that
harms or disregards the wellbeing of the system's owner. 

A legal pentest starts out different from a malicious attack in that the
tester gets the system's owner's permission first. The pentester can
optionally be given additional information about the system, to help guide
their attacks by avoiding time that would otherwise be wasted in gaining those
basic facts. 

\subsection{Acceptable Use Policies}

As an example, OSU's AUP contains the wording: 

\begin{quote}
    Entry into a system, including the network system,
    by individuals not specifically authorized (by
    group or personally) or attempts to circumvent the
    protective mechanisms of any University
    system are prohibited.
    (http://oregonstate.edu/senate/agen/2006/aupcurrent.pdf)
\end{quote}

Most acceptable use policies prohibit malicious attacks while permitting
authorized pentesting using a clause about {\bf authorized use}. 

Additionally, cloud hosting providers such as Amazon often have policies about
doing pentesting from servers they host, because pentesting attacks will
necessarily match the same profiles as illegal attacks\cite{aws}. 

\begin{quote}
    Permission is required for all penetration tests.
    To request permission, you must be logged into the AWS portal...
\end{quote}



\section{Jobs in the field} 

\section{Targets/Clients}

\section{Tools \& Techniques}

\subsection{Social engineering}

\subsection{SQL injection}

\subsection{XSS}

\subsection{JS injection}

\subsection{Privilege escalation}

\subsection{Known, unpatched vulnerabilities}

\section{What results might be found? }

\section{What actions would be taken by the company as a result?}

\begin{thebibliography}{1}
    \bibitem{whatispentest} https://www.sans.org/reading-room/analysts-program/PenetrationTesting-June06

    \bibitem{definition}
        http://searchsoftwarequality.techtarget.com/definition/penetration-testing

    \bibitem{aws} https://aws.amazon.com/security/penetration-testing/

    \bibitem{securitycurrent}
        http://www.securitycurrent.com/en/writers/mark-rasch/legal-issues-in-penetration-testing
    \bibitem{insidetrust} http://insidetrust.blogspot.com/2011/01/penetration-testing-permission.html
\end{thebibliography}
\end{document}
