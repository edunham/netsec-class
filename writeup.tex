\documentclass{article}
\usepackage{fullpage}
\title{Pentesting\\ECE478 Network Security}
\author{Dean Johnson \& Emily Dunham}
\linespread{1.5}

\begin{document}

\maketitle
\section{What's pentesting? }
    Penetration testing is the occupation of those security professionals who
attempt to gain access to a system using only the tools and knowledge available
to a malicious attacker, but with the system's owner's consent
\cite{whatispentest}. 

Pentesting includes automatic vulnerability scanning, manual testing, and a
variety of other techniques that mimic those which would be used by malicious
attackers. 

\subsection{Types of tests\cite{definition}}

{\bf Targeted testing} is a form of ``white box" testing in which testers work with
the client's internal teams who maintain the product being tested. This can
help simulate the level of knowledge and access available to an attacker such
as a disgruntled employee. 

{\bf External testing} explores how far an external attacker could get into a
client's systems by performing an attack from outside the company's network
and systems, trying to break in. External testing usually involves giving the
tester some special knowledge about the systems, which might not be available
to all malicious attackers. 

{\bf Internal testing} seeks ways that a disgruntled employee could escalate
the priveliges of standard system access in order to harm the customer's
systems. 

In {\bf blind testing}, the tester is given minimal information about the
network being tested. They will know the company's name, which can be used to
explore tech blogs, social networking sites, press releases, and any other
disclosures of information -- however, the tester knows no more than a
malicious attacker from outside the company would. 

For all of the above types of tests, it's assumed that all of the relevant
personnel inside the client company will be aware that the tests are being
performed. However, in {\bf double blind testing}, the client's teams who run
the systems being tested are not informed that the test is taking place.
Double blind is the only form of testing which can accurately predict the
client's response to a real attack, since in all other forms of testing people
are warned about the test and will be more vigilant during the testing period. 

\section{Pentesting Clients}

Any company or organization who could be harmed by a malicious attack on their
computer systems can benefit from penetration testing. Some of the most common
pentesting clients include banks, e-commerce sites, governments, corporations,
and militaries. 

Additionally, any company that builds or sells products that make security
claims, such as firewall programs or operating system code, can save itself
the embarrassment of some public vulnerability disclosures by having its
product pentested before release. Open source projects related to network
security, such as the router firmware project Tomato, generally get some level
of security testing from ordinary attacks against their users' deployments. 

\section{How can pentesting be done legally?}

The two key differences between pentesters and malicious attackers are that
the pentester attacks a system only with its owner's consent, and the
malicious attacker exploits discovered vulnerabilities in a way that
harms or disregards the wellbeing of the system's owner. 

A legal pentest starts out different from a malicious attack in that the
tester gets the system's owner's permission first. The pentester can
optionally be given additional information about the system, to help guide
their attacks by avoiding time that would otherwise be wasted in gaining those
basic facts. 

\subsection{Acceptable Use Policies}

As an example, OSU's AUP contains the wording: 

\begin{quote}
    Entry into a system, including the network system,
    by individuals not specifically authorized (by
    group or personally) or attempts to circumvent the
    protective mechanisms of any University
    system are prohibited.
    (http://oregonstate.edu/senate/agen/2006/aupcurrent.pdf)
\end{quote}

Most acceptable use policies prohibit malicious attacks while permitting
authorized pentesting using a clause about {\bf authorized use}. 

Additionally, cloud hosting providers such as Amazon often have policies about
doing pentesting from servers they host, because pentesting attacks will
necessarily match the same profiles as illegal attacks\cite{aws}. 

\begin{quote}
    Permission is required for all penetration tests.
    To request permission, you must be logged into the AWS portal...
\end{quote}



\section{Jobs in the field} 

Getting into the pen-testing / computer security industry is not too difficult.
It used to be that many 'hackers', as portrayed by the media anyways, were
black-hat, and often got into the industry by breaking systems without 
permission. However, in recent years their have been many improvements to our
current systems of reporting bugs. One such improvement has been bug bounties.

Bug bounty programs are set up by many well known companies, such as Facebook,
Google, Yahoo, and many more. These programs allow security researchers to try
and find security vulnerabilities in these companies systems and report them
for a financial reward. Before, people may have tried to find vulnerabilities
in systems for these companies without permission, and instead of being
rewarded, legal action was often taken against the hacker (even if the hacker
had good intentions). 

Another way of getting into the computer security industry is getting 
certifications, to prove you know what you're doing. However, getting real
clients yourself, or getting involved with a good consulting company usually
requires more than just certification; experience is the most valuable.

\section{Targets/Clients}

\section{Tools \& Techniques}

\subsection{Metasploit}

Metasploit is a collection of tools, the most popular of which is the free and
open source Metasploit Framework. The framework breaks down the task of
executing foreign code on a remote system into 4 simple steps, each of which
is configurable based on your use case. 

First, select an exploit. An exploit is simply a piece of code that takes
advantage of a bug in the target system. The Metasploit Framework comes with
over 900 exploits for Windows, Unix, Linux, and Mac systems. The framework can
optionally check whether the target system's version and specific
configuration are vulnerable to the selected exploit. All the exploit does is
allow you to run an arbitrary piece of code - choosing what that code does is
the next step.

Next, select a payload. The payload is the code that gets run on the target
system once the exploit succeeds. The payload could be as simple as a wall
message or fork bomb, or as complex as a VNC server. 

At this point, the combination of exploit and payload can be easily detected
by intrusion prevention systems. The third step of using metasploit is to
select an appropriate encoding technique to make the exploit and payload look
like benign data, or be ignored by the target's security systems entirely. 

Finally, execute the exploit. If you've chosen the exploit and payload
correctly, the payload code will run on the target system. To minimize
guesswork and wasted time, a pentester needs to know some basic information
about the operating system and network configuration of the target system. 

\subsection{NMAP}

Nmap is another tool in every pen-tester's arsenal. At it's most primitive
levels, it's a port scanner. However, it has so many more features than just
port scanning. By using nmap, you can find which computers are on a network,
what operating system (and version) those computers are running, the programs
(and their versions) running on those computers, and ports that are open on 
those computers as well. Using nmap, it makes it easy to do a system-wide 
assessment of the security on a network, and helps you lock down systems that 
are vulnerable.

A basic example of using nmap may look like this:

\begin{quote}
    nmap -v -A \textit{host(s)}
\end{quote}

What this does is run nmap with the verbose and application flags turned on,
against a given host. Verbose means it will return extra information for each 
of the scripts it runs, and -A ensures it returns applications running on the
target host. Since we enabled the -v option, it also returns version numbers
for each of those applications, so we can look for unpatched vulnerabilities
available for exploit on the server.

\subsection{Nessus}

Nessus is a (now) proprietary piece of software. For versions 2.2.11 and below,
they were GPL licenced, and the source code was available online. Nessus is
sort of like a type of swiss army knife of network security tools. Is has
a very unique pluggable architecture which allows you to take tools like
nmap, Hydra (a weak password checker) and Nitko (a cgi/.scripts checker) and
integrate them all together. Nesses also has some features unique to itself,
like checking server misconfigurations, denial of service attacks and
PCI DSS (Payment Card Industry Data Security Standard) Audits.

\subsection{Social engineering}

\section{Vulnerabilities}

\subsection{SQL injection}

The most common type of exploited vulnerability according to
\cite{mostcommonvulnerability} is SQL Injection. SQL Injection is a technique
that relies on formatting your input to a form that is sent to a database in
such a way that gives you either unauthorized permissions, or a view of more
data than intended. For example, if you have a login form, and know there's
an account named admin, you can specially format an input string to bypass
checking the password to the account if the SQL is not properly escaped.

We do this by making an educated guess at what the SQL query may look like,
then formatting it to do something other an intended. For example, a typical
login form probably has a SQL query that looks something like this:

\begin{quote}
    SELECT * FROM users WHERE username='\%s' AND password='\%s';
\end{quote}

If we then pass in a username that looks like "admin'; --", we effectively
make the statement look like this:

\begin{quote}
    SELECT * FROM users WHERE username='admin'; -- AND password='\%s';
\end{quote}

Since "--" escapes the rest of the statement, we then don't need a password,
and are authenticated as the admin user.

This obviously can have huge reprocutions. With statements that use SELECT and
display data from it, you can essentially format strings to get a dump of the
database. This can include passwords and other sensitive information, which
is why this vulnerability so critical to be aware of in your applications.

So, how do you prevent SQL Injection? The only thing you have to do is escape
your user inputs. This includes ignoring any characters that may be 'injected'
into your SQL statements that can give unintended results. So, as easy as it
is to make the mistake of leaving this open as a vulnerability, it's also
incredibly easy to patch.

\subsection{XSS}

Cross-site scripting (XSS) is another fairly common web application
vulnerability. Similarly to SQL Injection, what you essentially do is format
an input string that when displayed to another user, can cause harm to them.

For example, if we are tweeting out to the world and try to include <script>
in our tweet it escapes the <script> and doesn't ACTUALLY load a script
like the browser usually would when displaying your tweet to other users. This
can cause a lot of harm when a script is loaded that steals session tokens,
cookies, or other data stored in your browser that can allow an attacker to
impersonate you on other sites. 

\subsection{Privilege Escalation}

Another type of vulnerability, not specifically on the web application side of
things, is privilege escalation. This usually happens on the OS level and 
essentially means you gain access data which you are not supposed to have
access to. There are two types of privilege escalation.

The first is vertical privilege escalation. Vertical privilege escalation 
is when an attacker grants himself higher privileges than he is supposed to
have. An example of this is a developer granting himself root on a server 
where he is only is only supposed to have a basic level of access.

The other type of privilege escalation is horizontal. This is when an attacker
uses the same level of privileges he already has, but assumes the identify of
another user with the same privileges. For example, if a developer and system
administrator both have root on a box and the box is using a shared file
system for all users, th developer could "sudo su" as the system administrator,
and have access to all of his/her files, keys, and other sensitive information.

\subsection{Insecure File Creation}

Another system level vulnerability is insecure file creation. A broken 
application may try to create a temporary file in a public writeable directory
(e. g. /tmp) without forcing create-only. The attacker then can create a 
symlink pointing elsewhere (where the attacker does not have write
permission). The vulnerable application will overwrite the target
file (e. g. /home/someuser/.profile for normal users and a system files for 
root), giving them access to automatically run malicious code on the server.

\section{What results might be found? }

\section{What actions would be taken by the company as a result?}

\begin{thebibliography}{1}
    \bibitem{whatispentest} https://www.sans.org/reading-room/analysts-program/PenetrationTesting-June06

    \bibitem{definition}
        http://searchsoftwarequality.techtarget.com/definition/penetration-testing

    \bibitem{aws} https://aws.amazon.com/security/penetration-testing/

    \bibitem{securitycurrent}
        http://www.securitycurrent.com/en/writers/mark-rasch/legal-issues-in-penetration-testing
    \bibitem{insidetrust} http://insidetrust.blogspot.com/2011/01/penetration-testing-permission.html
    \bibitem{mostcommonvulnerability}
        http://www.sans.org/top25-software-errors/
\end{thebibliography}
\end{document}
